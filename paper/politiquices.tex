%
% File acl2020.tex
%
%% Based on the style files for ACL 2020, which were
%% Based on the style files for ACL 2018, NAACL 2018/19, which were
%% Based on the style files for ACL-2015, with some improvements
%%  taken from the NAACL-2016 style
%% Based on the style files for ACL-2014, which were, in turn,
%% based on ACL-2013, ACL-2012, ACL-2011, ACL-2010, ACL-IJCNLP-2009,
%% EACL-2009, IJCNLP-2008...
%% Based on the style files for EACL 2006 by
%%e.agirre@ehu.es or Sergi.Balari@uab.es
%% and that of ACL 08 by Joakim Nivre and Noah Smith

\documentclass[11pt,a4paper]{article}
\usepackage[hyperref]{acl2021}
\usepackage{times}
\usepackage{latexsym}
\renewcommand{\UrlFont}{\ttfamily\small}

% This is not strictly necessary, and may be commented out,
% but it will improve the layout of the manuscript,
% and will typically save some space.
\usepackage{microtype}

%\aclfinalcopy % Uncomment this line for the final submission

%\setlength\titlebox{5cm}
% You can expand the titlebox if you need extra space
% to show all the authors. Please do not make the titlebox
% smaller than 5cm (the original size); we will check this
% in the camera-ready version and ask you to change it back.

\newcommand\BibTeX{B\textsc{ib}\TeX}

\title{Politiquices: Exploring Political Relationships from Portuguese News Titles}

\author{First Author \\
  Affiliation / Address line 1 \\
  Affiliation / Address line 2 \\
  Affiliation / Address line 3 \\
  \texttt{email@domain} \\\And
  Second Author \\
  Affiliation / Address line 1 \\
  Affiliation / Address line 2 \\
  Affiliation / Address line 3 \\
  \texttt{email@domain} \\}

\date{}

\begin{document}
\maketitle
\begin{abstract}

\end{abstract}


\section{Introduction}
\label{sec:intro}


\section{Relationships}
\label{sec:rel}

% supports, defends,

% criticizes, opposses, accuses, attacks

% meet_together

% replaces


\subsection{Dataset}
\label{subsec:rel_data}

\subsubsection{Data Collection}
\label{subsubsec:rel_data_collec}


\subsubsection{Data Annotation}
\label{subsubsec:rel_data_annot}


\begin{table}[!h]
\centering
\begin{tabular}{lccr}
\hline \textbf{Relationship} & \textbf{Arg1,Arg2} & \textbf{Arg1,Arg2} & \textbf{Total} \\ \hline
opposes        &  &  \\
supports       &  &  \\
other          & - & - & \\
\hline
\end{tabular}
\caption{\label{font-table}  }
\end{table}

\subsection{Relationship Classifier}
\label{subsec:rel_classifier}


\subsection{Entity Linking}
\label{subsec:ent_linking}


\section{Crawling Data}
\label{sec:crawl}


\section{Triples Graph}
\label{sec:triples_graph}

%\bibliography{references}
%\bibliographystyle{acl_natbib}

\end{document}
