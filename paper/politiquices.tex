%
% File acl2020.tex
%
%% Based on the style files for ACL 2020, which were
%% Based on the style files for ACL 2018, NAACL 2018/19, which were
%% Based on the style files for ACL-2015, with some improvements
%%  taken from the NAACL-2016 style
%% Based on the style files for ACL-2014, which were, in turn,
%% based on ACL-2013, ACL-2012, ACL-2011, ACL-2010, ACL-IJCNLP-2009,
%% EACL-2009, IJCNLP-2008...
%% Based on the style files for EACL 2006 by
%%e.agirre@ehu.es or Sergi.Balari@uab.es
%% and that of ACL 08 by Joakim Nivre and Noah Smith

\documentclass[11pt,a4paper]{article}
\usepackage[hyperref]{acl2021}
\usepackage{times}
\usepackage{latexsym}
\renewcommand{\UrlFont}{\ttfamily\small}

% This is not strictly necessary, and may be commented out,
% but it will improve the layout of the manuscript,
% and will typically save some space.
\usepackage{microtype}
\usepackage{natbib}

\aclfinalcopy % Uncomment this line for the final submission

%\setlength\titlebox{5cm}
% You can expand the titlebox if you need extra space
% to show all the authors. Please do not make the titlebox
% smaller than 5cm (the original size); we will check this
% in the camera-ready version and ask you to change it back.

\newcommand\BibTeX{B\textsc{ib}\TeX}

\title{Politiquices: Exploring Support and Opposition Relationships \\in Archieved Portuguese Political News Headlines}

\author{David S. Batista \\
  Comtravo GmbH \\
  \texttt{david.batista@comtravo.com}
}

\date{}

\begin{document}
\maketitle
\begin{abstract}
News headlines within a political context often carry information evolving two political personalities.
In this work we analyzed hundreds of thousands of news headlines looking for those containing at least 
two political personalities with an entry on Wikidata and the presence of a support or opposition relationships 
from one political personality to another. This envoveld building different components: a knowledge base with
Wikidata entities and two classifiers one to detect the relationship and the second to detect the direction of
the relationship. The result is a graph represented as a RDF liking two political entities represented in Wikidata
through a news article which expresses a relationship of oppositon or support from one entity to the other.
\end{abstract}


\section{Introduction}
\label{sec:intro}


% https://www.emerald.com/insight/content/doi/10.1108/GKMC-07-2020-0098/full/html
% Understanding quotation extraction and attribution: towards automatic extraction of public figure’s statements for journalism in Indonesia


% supports, defends,
% criticizes, opposses, accuses, attacks

\section{Knowledge Base}
\label{sec_kb}

Since the target of our relationships are relevant portuguese political personalities we queried
Wikidata~\cite{MKGGB2018} to retrieve the page of several such personalities. This was achieved through a series
of queries issued through the Wikidata endpoint\footnote{\url{https://query.wikidata.org/}}, namely:

\begin{itemize}  
\item persons that are or were affiliated with portuguese political parties
\item portuguese persons born after 1935 whose profission is: judge, economist, lawyer, civil servant, politician, businessperson or banker
\item persons that had at least one position from a list of public office position (e.g., ministry, party leader, embassador, etc.)
\item a list of selected personalities that could not be retrived from the queries
\end{itemize}  

We download the Wikidata entry for each of the personalities retrieved from the 
queries above, using another endpoint\footnote{\url{https://www.wikidata.org/wiki/Special:EntityData?}},
and then select the label and alternative name properties for each entity, and together with the wikidata identifier
create an ElasticSearch index using these 3 fields.


\section{Crawling Data}
\label{sec:crawl}

The source for the news headlines was arquivo.pt, the portuguese web archive~\cite{SearchPastPWA2013}. We crawled news headlines through the arquivo.pt search API restricting the results to occurrences of names of political personalities gathered in Section~\ref{sec_kb} and targeting 45 selected .pt domains belonging diverse information sources, e.g.: online newspapers, television and radio stations websites and content aggregator portals. The CDX server interface allowed us to obtain the temporal limits (i.e., first and last crawl dates) of the archieved content for each domain, being the overall extremes between 1996 and 2019.

A second source of 







\section{Relatioships Dataset}
\label{subsubsec:rel_data_annot}


\begin{table}[!h]
\centering
\begin{tabular}{lccr}
\hline \textbf{Relationship} & \textbf{Arg1,Arg2} & \textbf{Arg2,Arg1} & \textbf{Total} \\ \hline
opposes        &   &     \\
supports       &   &     \\
other          & - & - & \\
\hline
\end{tabular}
\caption{\label{font-table}  }
\end{table}

\section{Classifiers}


\subsection{Entity Linking}
\label{subsec:ent_linking}


\subsection{Relationship Classifier}
\label{subsec:rel_classifier}

\subsection{Relationship Direction Classifier}
\label{subsec:rel_direction}






\section{Triples Graph}
\label{sec:triples_graph}






\bibliography{references}
\bibliographystyle{acl_natbib}

\end{document}
